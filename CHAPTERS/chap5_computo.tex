\chapter{Stima dei costi di costruzione}
Per ultima analisi viene riportata una stima dei costi degli elementi progettati. 
Per quanto riguarda il progetto base di cui al paragrafo \ref{cap:ProgettoCondotte} è stato svolto un computo metrico estimativo andando ad analizzare tutte le lavorazioni necessarie utilizzando il prezzario della Provincia Autonoma di Trento aggiornato all'anno 2021. 
Tale computo è visibile nella sua interezza tra gli allegati riportati in appendice e in particolare a pagina \pageref{appendix:computo}.
Di seguito è invece riportato un resoconto con i totali parziali di ciascuna categoria.
\begin{table}[H] 
    \centering
    \caption{Resoconto dei totali parziali del computo metrico estimativo riguardo il progetto di partenza}
    \label{tab:computoRiepilogo}
    \begin{tabular}{lS[table-format=6.2]}
        \toprule
		\multicolumn{1}{l}{Lavorazione} & \multicolumn{1}{c}{Valore \si{[€]}}     \\
        \midrule
		Totale condotte                 & 55236.08  \\
		Totale pozzetti tipo Europa     & 65677.20  \\
		Totale movimento terre          & 46038.00  \\
		\B  Totale                      & \B 166951.28 \\
        \bottomrule
\end{tabular}%
\end{table}

Nel caso delle vasche di laminazione è stata invece eseguita una stima tramite confronto del valore tra computi metrici già eseguiti. 
Rapportando il volume totale delle vasche pari all'altezza di \SI{1.50}{\metre} per l'area riportata precedentemente nella tabella \ref{tab:iterazioni} e considerando la loro costruzione con calcestruzzo gettato in opera, si è pervenuti a una stima compresa tra \SI{200000} e \SI{250000}{€}.

È riportata infine una tavola con la rappresentazione dei principali elementi costituenti la rete di drenaggio a pagina \pageref{appendix:Tavola1}.