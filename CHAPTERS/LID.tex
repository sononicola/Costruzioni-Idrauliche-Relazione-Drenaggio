\section{Progetto con vasche e lid}
\phantomsection
Viene ora valutata l'aggiunta di sistemi LID per capirne l'effetto benefico che possono portare al progetto fin qui visto. 
Questi sistemi vengono infatti detti a basso impatto ambientale (Low Impact Development, LID) e rappresentano l’insieme di strategie e tecniche che vengono utilizzate nella gestione sostenibile delle acque meteoriche. 
Il loro scopo è quello di controllare e trattenere una buona parte delle acque meteoriche tramite processi pressoché naturali laminando le portate o mantenendo il regime idrico (concetto di invarianza idraulica) presente prima dell'urbanizzazione.

Esistono varie tipologie di LID: in questo progetto si  è deciso di applicare celle di bio-ritenzione, tetti verdi e pavimentazioni permeabili.
La prima tipologia riguardano aree con uno strato inferiore di ghiaia che funge da drenaggio o da immagazzinamento delle acque delle aree circostanti.
Sopra a questo primo strato si ha del terreno per favorire la nascita e crescita di alberature di media e piccola grandezza.
Per quanto riguarda i tetti verdi questi vengono considerati una variante delle celle di bio-ritenzione con uno strato di terra meno alto e con strati sintetici, fatti di guaine, per una maggiore capacità drenante.
Infine le pavimentazioni permeabili consistono in aree composte da uno strato cementizio o di asfalto drenante, dove l'acqua può filtrare nello strato sottostante che funge da immagazzinamento e da drenaggio.

\subsection{Procedimento e progetto}
Al fine di applicare i sistemi LID si deve dapprima scegliere le aree in cui è opportuna la loro installazione, per poi andare a definire tutti quei parametri geotecnici e costruttivi atti a soddisfare le caratteristiche dei manufatti che si vogliono realizzare.
Precedentemente a questo occorre inoltre analizzare il suolo dei sottobacini dello stato di fatto iniziale in modo da conoscere l'estensione dell'area su cui avranno luogo i LID, la loro dimensione pianificabile, e la percentuale di saturazione e di impermealizzione del suolo stesso. 
Essendo molto numerosi i parametri da inserire in SWMM, si riporta in appendice a pagina \pageref{appendix:codiceLID} direttamente il codice di input necessario alla loro elaborazione nel programma. 
Si riportano in aggiunta delle schermate dei tre sistemi progettati atte a far corrispondere il codice con il parametro del LID.

Le celle di bio-ritenzione si sono posizionate lungo tutte le canalette presenti nel quartiere. 
È prevista una vegetazione densa al loro interno, con una ampia capacità di immagazzinamento senza possibilità di drenaggio nella rete principale.

I tetti verdi vengono posizionati nella maggior parte degli edifici preesistenti, ove ne è possibile l'installazione. 

Per quanto riguarda le aree di pavimentazione permeabile si sono poste in corrispondenza del parcheggio Monte Baldo e dell'ampia area di manovra a nord del parcheggio stesso, in modo da non impattare la funzione della zona stessa e mantenerla ad uso parcheggio.
A livello di pavimentazione è prevista l'installazione di blocchetti in calcestruzzo con attorno la possibilità di crescita della vegetazione con una percentuale del \SI{60}{\percent}.

\subsection{Considerazioni finali}
Eseguita l'analisi si è voluto confrontare i diversi stati progettuali al fine di valutarne la loro  efficacia. 
A tale scopo è stata svolta un'analisi anche con solo la presenza dei LID e senza le vasche di laminazione.

Con tali configurazioni della rete è stato rifatto il procedimento descritto all'inizio di questo capitolo per la riprogettazione delle condotte. 
In figura \ref{fig:ConfrontoLID} è rappresentato l'andamento del deflusso ai recapiti finali nelle quattro casistiche.
Si è notato come l'inserimento dei LID porti, soprattutto nelle aree altamente impermeabili poste a sud e defluenti nel terzo recapito finale, ad una riduzione del deflusso totale. 
Questo comporta inoltre la possibilità di diminuire la dimensione delle condotte e il volume delle vasche. Le condotte 1, 2, 15, 28 possono infatti avere un diametro di \SI{10}{\centi\metre} minore rispetto a quelli elencati nella tabella \ref{tab:Diametri_conduct-mod} riportata precedentemente a pagina \pageref{tab:Diametri_conduct-mod}. 
Le prime due vasche potrebbero ridursi di circa \SI{30}{\metre\cubed} mentre la terza di oltre \SI{120}{\metre\cubed}, pari a $1/3$ dell'originale.

\begin{figure}[htbp]
    \centering
    \begin{tikzpicture}
        \begin{axis}[
            restrict x to domain=-0:4,
            height=6.5cm,
            width=15cm,
            grid=major,
            xlabel=Tempo trascorso dall'inizio della precipitazione \si{[\hour]},
            ylabel=Deflusso  \si{[\litre\per\second]},
            xtick = {0,0.5,...,4},
            title= \emph{Recapito finale 1},
            /pgf/number format/.cd,
            use comma,
            1000 sep={\,}
        ]
        \addplot +[mark=none,style=solid,color=red] table[x index=0,y index=1,header=false] {IMG/LID-Inflow/PreVasca_Attenuazione1.txt};
        \addplot +[mark=none,style=solid,color=blue] table[x index=0,y index=1,header=false] {IMG/LID-Inflow/SoloLID_Inflow1.txt};
        \addplot +[mark=none,style=solid,color=cyan] table[x index=0,y index=1,header=false] {IMG/LID-Inflow/Vasca_Inflow1.txt};
        %\addplot +[mark=none,style=solid,color=green] table[x index=0,y index=1,header=false] {IMG/LID-Inflow/LIDvasca_Inflow1.txt};
        \addplot +[mark=none,style=solid,color=orange] table[x index=0,y index=1,header=false] {IMG/LID-Inflow/LIDvascamod_Inflow1.txt};
        
        \legend{Solo condotte, + LID, + Vasche, + Vasche e LID, }    
        \end{axis}
    \end{tikzpicture}

    \vspace{.5cm}
    %
    \begin{tikzpicture}
        \begin{axis}[
            restrict x to domain=-0:4,
            height=6.5cm,
            width=15cm,
            grid=major,
            xlabel=Tempo trascorso dall'inizio della precipitazione \si{[\hour]},
            ylabel=Deflusso  \si{[\litre\per\second]},
            xtick = {0,0.5,...,4},
            title= \emph{Recapito finale 2},
            /pgf/number format/.cd,
            use comma,
            1000 sep={\,}
        ]
        \addplot +[mark=none,style=solid,color=red] table[x index=0,y index=1,header=false] {IMG/LID-Inflow/PreVasca_Attenuazione2.txt};
        \addplot +[mark=none,style=solid,color=blue] table[x index=0,y index=1,header=false] {IMG/LID-Inflow/SoloLID_Inflow2.txt};
        \addplot +[mark=none,style=solid,color=cyan] table[x index=0,y index=1,header=false] {IMG/LID-Inflow/Vasca_Inflow2.txt};
        %\addplot +[mark=none,style=solid,color=green] table[x index=0,y index=1,header=false] {IMG/LID-Inflow/LIDvasca_Inflow2.txt};
        \addplot +[mark=none,style=solid,color=orange] table[x index=0,y index=1,header=false] {IMG/LID-Inflow/LIDvascamod_Inflow2.txt};
        
        \legend{Solo condotte, + LID, + Vasche, + Vasche e LID  } 
        \end{axis}
    \end{tikzpicture}

    \vspace{.5cm}
    %
    \begin{tikzpicture}
        \begin{axis}[
            restrict x to domain=-0:4,
            height=6.5cm,
            width=15cm,
            grid=major,
            xlabel=Tempo trascorso dall'inizio della precipitazione \si{[\hour]},
            ylabel=Deflusso  \si{[\litre\per\second]},
            xtick = {0,0.5,...,4},
            title= \emph{Recapito finale 3},
            /pgf/number format/.cd,
            use comma,
            1000 sep={\,}
        ]
        \addplot +[mark=none,style=solid,color=red] table[x index=0,y index=1,header=false] {IMG/LID-Inflow/PreVasca_Attenuazione3.txt};
        \addplot +[mark=none,style=solid,color=blue] table[x index=0,y index=1,header=false] {IMG/LID-Inflow/SoloLID_Inflow3.txt};
        \addplot +[mark=none,style=solid,color=cyan] table[x index=0,y index=1,header=false] {IMG/LID-Inflow/Vasca_Inflow3.txt};
        %\addplot +[mark=none,style=solid,color=green] table[x index=0,y index=1,header=false] {IMG/LID-Inflow/LIDvasca_Inflow3.txt};
        \addplot +[mark=none,style=solid,color=orange] table[x index=0,y index=1,header=false] {IMG/LID-Inflow/LIDvascamod_Inflow3.txt};

        \legend{Solo condotte, + LID, + Vasche, + Vasche e LID  } 
        \end{axis}
    \end{tikzpicture}
    %
    \caption{Confronto del deflusso allo sbocco della rete pre e post l'installazione dei LID}
    \label{fig:ConfrontoLID}
\end{figure}