\chapter{Caratteristiche pluviometriche dell’area di studio}\label{cap:pluviometriche}
\section{Curve di possibilità pluviometrica}
Dopo che si è eseguito un primo inquadramento della zona si procede con le elaborazioni dei dati dei massimi annuali degli scrosci e delle precipitazioni orarie ricavate dalla stazione pluviometrica di Laste a Trento.
Attraverso queste elaborazioni, svolte attraverso il metodo di BOOOOO(momenti o minimi quadrati o massima verosimiglianza) si pone l'obiettivo di determinare le curve di possibilità pluviometrica (Fig.2) e i propri parametri (Tab.1 (a[mm/h alla n];n)) con diversi tempi di ritorno ($T_r$). 
Per la relazione si considera un tempo di ritorno pari a \si{25} anni. Nella Fig.2, dove si confrontano le CPP degli scrosci e delle precipitazioni orarie, si evince che si ottiene una maggiore altezza di precipitazione, dovuta agli scrosci, per le prime tre ore e mezza e poi le precipitazioni orarie superano gli scrosci. 
Quindi per la seguente relazione si prenderanno in considerazione gli scrosci. 
Si riporta la Tab.1 dei parametri con $T_r$=\si{25} anni per gli scrosci e le precipitazioni orarie:

tab.1 fig.2

per figure grafici CPP 
\begin{equation}
    h = a(T_r) \, t_p ^{n}
\end{equation}