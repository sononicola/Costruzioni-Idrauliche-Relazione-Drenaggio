\chapter{Verifiche alle condotte}
% riempimento
% velocità
% i f i feom
% tau

Il riempimento della condotta $G_\text{cond.}$ deve risultare 
\begin{equation}
    \SI{50}{\percent} \lessapprox G_\text{cond.} \lessapprox\SI{75}{\percent}
\end{equation}


\begin{equation}
    \SI{0.5}{\metre\per\second} < V <  \SI{5}{\metre\per\second}
\end{equation}

Criterio di autopulizia
\begin{equation}
    \tau = \gamma \, R_H \, i_F > \SI{2}{\pascal}
\end{equation}
dove $\gamma$ è il peso specifico dell'acqua pari a \SI{1000}{\newton\per\metre\cubed}, $R_H$ è il raggio idraulico calcolato con la formula di BOH NUM e $i_f$ è la pendenza del fondo vista prima TAB.  
\begin{align}
    R_H &= \frac{D}{4} \, \frac{1 - \sin(\vartheta)}{\vartheta} \\
    \vartheta &= 2 \, \arccos(1 - G_\text{cond.})
\end{align}

\TabellaVerificheLinkFLow{Progetto con aggiunta dei soli LID -- Verifiche di massima velocità, riempimento condotta e del criterio di autopulizia}{tab:LinkFlow_Verifiche-MOD-LID}{IMG/LinkFlow-Verifiche-MOD-LID.tex}
 